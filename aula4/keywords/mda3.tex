\documentclass{beamer}
% \documentclass[handout]{beamer}

\usetheme{dcc}

\usepackage[utf8]{inputenc}
\usepackage[american]{babel}
\usepackage{url}

\newcommand{\ie}{i.\,e.}
\newcommand{\Ie}{I.\,e.}
\newcommand{\eg}{e.\,g.}
\newcommand{\Eg}{E.\,g.}

\title{Mineração de Dados Aplicada}
\subtitle{Simple but Powerful Text-Processing Commands}

\author{Loïc Cerf}

\date{August, 31$^{\text{st}}$ 2015}

\institute
{
  \scriptsize
  DCC -- ICEx -- UFMG
  \bigskip\\
  \includegraphics[width=.334113288\linewidth]{images/dcc}\hfill\includegraphics[width=.5\linewidth]{images/ufmg}
}

\subject{Simple but Powerful Text-Processing Commands}

\keywords{text files, commands, POSIX}

\hypersetup{pdfpagemode=FullScreen}

\AtBeginSection[]
{
  \begin{frame}<beamer>
    \frametitle{Outline}
    \tableofcontents[currentsection]
  \end{frame}
}

\begin{document}

\begin{frame}[plain]
  \titlepage
\end{frame}

\section*{Summary of the last session}

\begin{frame}
  \frametitle{Data exploration}
  \begin{itemize}
  \item Attributes can be categorized into four types: nominal,
    ordinal, interval-scaled and ratio-scaled;
  \item The type tells what operations, statistics and data mining
    algorithms are applicable;
  \item Data usually are incomplete, inconsistent, with some
    exceptions, uncertain/noisy or even plainly wrong;
  \item Basic statistics and visualizations help a lot in
    understanding the data; looking at specific objects too;
  \item Correlation does not imply causation.
  \end{itemize}
\end{frame}

\section*{Simple but powerful text-processing commands}

\begin{frame}
  \frametitle{Unix philosophy}
  \begin{block}{Unix philosophy}
    Doug McIlroy (inventor of Unix pipes). In \emph{A Quarter-Century
      of Unix} (1994):\\
    \begin{quotation}
      Write programs that do one thing and do it well. Write programs
      to work together. Write programs to handle text streams, because
      that is a universal interface.
    \end{quotation}
  \end{block}
\end{frame}

\begin{frame}
  \frametitle{Utilities from the 70s}
  The Unix operating system came with several text-processing commands
  that are still very useful today. Specific, these commands are very
  efficient. The GNU project, started in 1984, has improved them a
  great deal (\eg, additional options). The original commands are part
  of some POSIX standards.

  \vfill
  \pause

  Most Free operating systems are POSIX-compliant: GNU/Linux, BSD,
  illumos, Haiku, etc. Mac OS X is too. Windows is not but Cygwin is a
  good compatibility layer.
\end{frame}

\begin{frame}
  \frametitle{The shell}
  The \emph{shell} interprets every command fired in a terminal or in
  an executable file whose first line indicates the command-line
  interpreter. Modern shells (\texttt{bash}, \texttt{dash},
  \texttt{zsh}, etc.) extend the original UNIX shell, \texttt{sh},
  which is normalized in POSIX.

  \vfill

  \begin{exampleblock}{A 2-line shell script}
    \texttt{\#!/bin/sh}\\
    ~\\
    \texttt{\# Get started with the MDA exercises}\\
    \texttt{mv data data-`date +\%m\%d` 2> /dev/null}\\
    \texttt{wget \url{dcc.ufmg.br/~lcerf/data.tar.xz} -O - | tar -xJ}
  \end{exampleblock}
\end{frame}

\begin{frame}
  \frametitle{Geeks and repetitive tasks}
  \includegraphics[width=\linewidth]{img/repetitive-tasks}
\end{frame}

\begin{frame}
  \frametitle{Standard I/O}
  Almost all POSIX text processing commands:
  \begin{itemize}
  \item process the input line by line;
  \item read, by default, the standard input (the keyboard if not
    redirected);
  \item write, by default, on the standard output (the terminal if not
    redirected).
  \end{itemize}

  \vfill
  \pause

  \texttt{<}/\texttt{>} redirects the standard I/O from/to a file. A
  pipe binds an output stream to an input stream. It can bear a name
  (in argument of \texttt{mkfifo}) but most workflows only need the
  unnamed pipe, \texttt{|}, that redirects the standard output of the
  command on the left to the standard input of the command on the
  right.
\end{frame}

\begin{frame}
  \frametitle{Getting the data}
  \texttt{\$ wget \url{dcc.ufmg.br/~lcerf/data.tar.xz} -O - | tar -xJ}

  \vfill
  \pause

  \texttt{wget} and \texttt{tar} are two GNU commands. Like all GNU
  commands:
  \begin{itemize}[<+->]
  \item the \texttt{man} command (\eg, \texttt{man wget}) gives their
    specifications;
  \item the \texttt{info} command (\eg, \texttt{info wget}) often
    provides more detailed explanations, examples of use, etc.;
  \item long options are prefixed with \texttt{{-}{-}}, short (\ie,
    one letter) options with \texttt{-} and can be grouped (\eg,
    \texttt{-xJ});
  \item options can have an argument and \texttt{-} means the standard
    input (\texttt{/dev/stdin}) or the standard output
    (\texttt{/dev/stdout});
  \item the unnamed pipe, \texttt{|}, redirects the standard output of
    the command on the left to the standard input of the command on
    the right.
  \end{itemize}
\end{frame}

\begin{frame}
  \frametitle{Reading a large text file}
  Your favorite text editor (Vim or Emacs?) loads the entire file in
  main memory what raises problems if it weights gigabytes or more.

  \vfill
  \pause

  The solution is named \texttt{less}. It is the viewer for
  \texttt{man} pages.

  \vfill

  A few commands inside \texttt{less}: Page-up/down, R (repaint), F
  (follow), [0-9]+ (scroll that many lines), / (search forwards for a
  regexp), ? (search backwards for a regexp), q (quit).

  \vfill
  \pause

  \begin{exampleblock}{Exercise}
    Find, with \texttt{less}, the IP address of the first Brazilian
    visitor after the $100^{\text{th}}$ line of
    \texttt{DistroWatch/20100428/debian}.
  \end{exampleblock}
\end{frame}

\begin{frame}
  \frametitle{Printing the beginning/end of a file}
  Do not test your scripts on the whole dataset!

  \vfill

  \texttt{head} outputs the head of a file; \texttt{tail} its
  tail.

  \vfill

  A few options: -[0-9]+ tunes the number of lines (10 by default), -f
  follows the appended data.

  \vfill
  \pause

  \begin{exampleblock}{Exercise}
    Print the lines 5 to 15 of one of the file in
    \texttt{DistroWatch.com}'s logs.
  \end{exampleblock}
\end{frame}

\begin{frame}
  \frametitle{Concatenating files}
  The \texttt{cat} command concatenates files and prints them.

  \vfill
  \pause

  \begin{exampleblock}{Exercise}
    Concatenate the files related to the visits to the Ubuntu page.
  \end{exampleblock}
\end{frame}

\begin{frame}
  \frametitle{Translating characters}
  Given, in argument, two sets of characters, the \texttt{tr} command
  reads the standard input and outputs it after replacing every
  character in the first set with the one at the same position in the
  second set. The last character in the second set is considered
  repeated up to the size of the first set.

  \vfill

  A few options: -c specifies the first set as the complement of the
  provided characters -s squeezes the repetitions of the characters in
  the first set, -d deletes those characters.
\end{frame}

\begin{frame}
  \frametitle{Basic reformatting with \texttt{tr}}
  \begin{exampleblock}{Exercise}
    From one of the file in \texttt{DistroWatch.com}'s logs:
    \begin{enumerate}
    \item change the delimiters into spaces;
    \item make the country codes lower cased.
    \end{enumerate}
  \end{exampleblock}

  \vfill

  Notice that the shell (\ie, the command interpreter) understands
  some characters in a special way. For instance, the command line is
  broken w.r.t. whitespaces, \texttt{\$} precedes a shell variable
  whose content is to be read, etc. To preserve the literal meaning:
  \begin{itemize}
  \item a backslash can precede every special character;
  \item a (portion of an) argument can be enclosed with single quotes
    (with double quotes to still let the shell interpret \texttt{\$}
    and \texttt{`}).
  \end{itemize}
\end{frame}

\begin{frame}
  \frametitle{Counting lines, words or characters}
  The \texttt{wc} command counts the number of lines (option -l), of
  words (option -w) or of characters (option -m).

  \vfill
  \pause

  \begin{exampleblock}{Exercise}
    In \texttt{DistroWatch.com}'s logs, what are the number of visits
    per day to the Ubuntu page. Is there something special (compare
    with other distributions)?
  \end{exampleblock}
\end{frame}

\begin{frame}
  \frametitle{Selecting fields}
  The \texttt{cut} command selects fields. \texttt{cut} considers that
  there is an empty field between two subsequent delimiters.

  \vfill

  -d specifies the delimiter, -f specifies the fields \texttt{cut}
  must keep.

  \vfill
  \pause

  \begin{exampleblock}{Exercise}
    From one of the file in \texttt{DistroWatch.com}'s logs, get rid
    of the IP addresses.
  \end{exampleblock}
\end{frame}

\begin{frame}
  \frametitle{Pasting}
  The \texttt{paste} command concatenates the lines of the input files
  in the order they are given.

  \vfill

  -d specifies the delimiter.
\end{frame}

\begin{frame}
  \frametitle{Comparing}
  \texttt{comm} compares the lines of two \emph{sorted} files and
  outputs those unique to the first file (first column), to the second
  file (second column) and those common to both files (third column).

  \vfill

  -1, -2 and -3 respectively remove the first, second and third column
  from the output.
\end{frame}

\begin{frame}
  \frametitle{Joining}
  \texttt{join} joins the lines of two files. Both files must be
  \emph{sorted} according to their join field.

  \vfill

  A few options: -1 NUM (set the field on which the join is done in
  the first file), -2 NUM (in the second file), -j (in both files), -a
  {1,2} (prints unpairable lines from the specified file too), -i
  (ignore case), -t CHAR (set delimiter).
\end{frame}

\begin{frame}
  \frametitle{Sorting}
  The \texttt{sort} command sorts a text file. It depends on the
  LC\_ALL variable (you may want LC\_ALL=C).

  \vfill

  A few options: -r (reverse sort), -f (ignore case), -n (numeric
  sort), -c (check if sorted), -k POS1[,POS2] (sort according to the
  fields in the interval), -t (set the field separator), -u (remove
  duplicates), -m (merge already sorted files).

  \vfill
  \pause

  \begin{exampleblock}{Exercise}
    In \texttt{DistroWatch.com}'s logs:
    \begin{enumerate}
    \item How many different countries are there?
    \item On the 29th, what are the countries where the visitors have
      loaded some page but not that of Ubuntu?
    \end{enumerate}
  \end{exampleblock}
\end{frame}

\begin{frame}
  \frametitle{Reporting or omitting repeated lines}
  The \texttt{uniq} command removes \emph{adjacent} repeated lines.

  \vfill

  A few options: -c (give the counts of repetitions), -d (only print
  repeated lines), -u (only print unique lines) -i (ignore case).

  \vfill
  \pause

  \begin{exampleblock}{Exercise}
    In \texttt{DistroWatch.com}'s logs, rank the countries by their
    numbers of accesses to the index page during the three days.
  \end{exampleblock}
\end{frame}

\section*{Homework}

\begin{frame}
  \frametitle{Exercise}
  \begin{alertblock}{Exercise for next Wednesday}
    Given a list of keywords and a text, use the commands presented so
    far to tell how many times each of these keywords occurs in the
    text.
  \end{alertblock}
\end{frame}

\section*{}

\begin{frame}
  \frametitle{License}
  \begin{block}{\copyright 2012--2015 Loïc Cerf}
    \includegraphics[height=1.5ex]{img/licenses/cc-by-sa}\hskip1pt
    These slides are licensed under the Creative Commons
    Attribution-ShareAlike 4.0 International License.
  \end{block} 
\end{frame}

\end{document}